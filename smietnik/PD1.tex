\documentclass[10pt]{article}
\usepackage{polski}
\usepackage[utf8]{inputenc}
\usepackage[polish]{babel}
\usepackage[left=2.5cm, right=2.5cm, top=2cm, bottom=2.8cm]{geometry}

\begin{document}
\author{Grupa 9}
\title{Analiza przeżycia -- praca domowa 1}
\maketitle

Plik zawiera dane dla 686 chorych na raka piersi oraz następujące zmienne:
\begin{itemize}
\item \texttt{id} -- identyfikator chorej -- pomijamy
\item \texttt{meno} -- wskaźnik menopauzy (1 –- nie, 2 -- tak) 
\item \texttt{horm} -- wskaźnik użycia terapii hormonalnej (1 –- nie, 2 -- tak) 
\item \texttt{prog} -- wskaźnik receptorów progesteronu (0 –- ujemny, 1 –- dodatni) 
\item \texttt{estr} -- wskaźnik receptorów estrogenu (0 –- ujemny, 1 –- dodatni) 
\item \texttt{grade} -- stopień zróżnicowania komórek nowotworu (1 –- wysoki, 2 -- średni, 3 -- niski) 
\item \texttt{size} -- wielkość guza (w cm) -- zmienna ciągła
\item \texttt{nodes} -- liczba węzłów chłonnych z przerzutami nowotworu -- zmienna ciągła
\item \texttt{rectime} -- czas przeżycia bez nawrotu choroby (dni) 
\item \texttt{censrec} -- wskaźnik zdarzenia (0 –- cenzurowanie, 1 -- zgon lub nawrót choroby)
\end{itemize}

Interesuje nas pytanie, które zmienne mają wpływ na czas przeżycia bez nawrotu choroby. \\\\

Proponujemy model proporcjonalnych hazardów (PH). \\
Sprawdzamy spełnianie założeń modelu dla każdej ze zmiennych (poza ciągłymi) poprzez narysowanie krzywych Kaplana-Meiera oraz wykresów transformacji log(-log) krzywych przeżycia. \\
Krzywe dla rożnych poziomów zmiennej \texttt{meno} przecinają się, co oznacza niespełnianie założeń modelu PH.
Dla pozostałych zmiennych krzywe nie przecinają się oraz wykresy transformacji możemy uznać za nieodstające od równoległych. \\\\
Proponujemy więc model proporcjonalnych hazardów warstwowany względem zmiennej \texttt{mono}. \\\\
W celu sprawdzenia, czy zmienne ciągłe nie powinny zostać przekształcone przed wprowadzeniem ich do~modelu, wykonujemy wykresy reszt martyngałowych pustego modelu od każdej z tych zmiennych. Wykresy sugerują przekształcenie zmiennych poprzez logarytm, więc dla potwierdzenia naszych przypuszczeń rysujmy jeszcze wykresy reszt od logarytmów zmiennych i obserwujemy, że wykresy są teraz bliższe liniowym. Wprowadzamy więc do modelu zmienne ciągłe przekształcone logarytmem.  \\\\
+ testy logrank dopisać ?? \\\\
W celu formalnej oceny spełnienia założeń PH przeprowadzamy test Shoenfelda \\
+ wyniki testu dopisac ?? \\
Zarówno globalny test, jak i wszystkie pojedyncze testy mają p-wartość większą od 0.05, więc nie mamy podstaw do~odrzucenia hipotezy postulującej, że współczynniki przy tych zmiennych są stałe w czasie.  \\
Przyjrzyjmy się również wykresom skalowanych reszt Schoenfelda dla każdej zmiennej w modelu od czasu i~dopasujmy krzywe do danych. Dołączone granice dla 95\% obszaru ufności nie dają podstawy do odrzuceni hipotezy, że krzywa nie różni się od horyzontalnej.  \\
Zarówno wykresy, jak i wyniki testu nie sugerują odstępstw od założenia PH. \\\\
W stworzonym modelu zmiennymi istotnymi statystycznie na poziomie istotności 0.05 są: \texttt{horm}, \texttt{prog}, log(\texttt{nodes}). Współczynniki modelu przy tych zmiennych wynoszą odpowiednio: -0.39, -0.69, 0.49. Oznacza to, że jeśli została użyta terapia hormonalna, to hazard zgonu lub nawrotu choroby zmienia się exp(-0.39)=0.68 raza w stosunku do nie użycia terapii hormonalnej (gdy wszystkie inne zmienne są takie same). Jeśli wskaźnik receptorów progesteronu zmienia się z ujemnego na dodatni, to hazard zmieni się o exp(-0.69)=0.5 raza, natomiast gdy logarytm liczby węzłów chłonnych z przerzutami nowotworu wzrośnie o 1, to hazard zwiększy się exp(0.49)=1.63 raza. \\\\
Podsumowanie modelu jest następujące:\\
+ podsumowanie wkleić ??\\\\
Testy Likelihood, Wald i Score dają p-wartość mniejszą od 0.05, co oznacza, że odrzucamy hipotezę o tym, że model pusty jest równie dobrze dopasowany jak dopasowany model na korzyść hipotezy, że dopasowany model jest od niego lepszy. <- poprawić. \\\\
W celu sprawdzenia  dopasowania modelu rysujemy wykresy reszt dewiancji/martyngałowych od liniowej kombinacji zmiennych/indeks.  
Na żadnym z wykresów nie można wskazać wyraźnie odstających reszt. Wykres reszt dewiancji od indeksu można uznać za symetryczny, pozostałe niekoniecznie, co sugeruje, że dopasowany model nie jest perfekcyjny. 

\end{document}