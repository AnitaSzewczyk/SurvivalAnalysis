\documentclass[]{article}
\usepackage{lmodern}
\usepackage{amssymb,amsmath}
\usepackage{ifxetex,ifluatex}
\usepackage{fixltx2e} % provides \textsubscript
\ifnum 0\ifxetex 1\fi\ifluatex 1\fi=0 % if pdftex
  \usepackage[T1]{fontenc}
  \usepackage[utf8]{inputenc}
\else % if luatex or xelatex
  \ifxetex
    \usepackage{mathspec}
    \usepackage{xltxtra,xunicode}
  \else
    \usepackage{fontspec}
  \fi
  \defaultfontfeatures{Mapping=tex-text,Scale=MatchLowercase}
  \newcommand{\euro}{€}
\fi
% use upquote if available, for straight quotes in verbatim environments
\IfFileExists{upquote.sty}{\usepackage{upquote}}{}
% use microtype if available
\IfFileExists{microtype.sty}{%
\usepackage{microtype}
\UseMicrotypeSet[protrusion]{basicmath} % disable protrusion for tt fonts
}{}
\usepackage[margin=1in]{geometry}
\usepackage{color}
\usepackage{fancyvrb}
\newcommand{\VerbBar}{|}
\newcommand{\VERB}{\Verb[commandchars=\\\{\}]}
\DefineVerbatimEnvironment{Highlighting}{Verbatim}{commandchars=\\\{\}}
% Add ',fontsize=\small' for more characters per line
\newenvironment{Shaded}{}{}
\newcommand{\KeywordTok}[1]{\textcolor[rgb]{0.00,0.44,0.13}{\textbf{{#1}}}}
\newcommand{\DataTypeTok}[1]{\textcolor[rgb]{0.56,0.13,0.00}{{#1}}}
\newcommand{\DecValTok}[1]{\textcolor[rgb]{0.25,0.63,0.44}{{#1}}}
\newcommand{\BaseNTok}[1]{\textcolor[rgb]{0.25,0.63,0.44}{{#1}}}
\newcommand{\FloatTok}[1]{\textcolor[rgb]{0.25,0.63,0.44}{{#1}}}
\newcommand{\CharTok}[1]{\textcolor[rgb]{0.25,0.44,0.63}{{#1}}}
\newcommand{\StringTok}[1]{\textcolor[rgb]{0.25,0.44,0.63}{{#1}}}
\newcommand{\CommentTok}[1]{\textcolor[rgb]{0.38,0.63,0.69}{\textit{{#1}}}}
\newcommand{\OtherTok}[1]{\textcolor[rgb]{0.00,0.44,0.13}{{#1}}}
\newcommand{\AlertTok}[1]{\textcolor[rgb]{1.00,0.00,0.00}{\textbf{{#1}}}}
\newcommand{\FunctionTok}[1]{\textcolor[rgb]{0.02,0.16,0.49}{{#1}}}
\newcommand{\RegionMarkerTok}[1]{{#1}}
\newcommand{\ErrorTok}[1]{\textcolor[rgb]{1.00,0.00,0.00}{\textbf{{#1}}}}
\newcommand{\NormalTok}[1]{{#1}}
\ifxetex
  \usepackage[setpagesize=false, % page size defined by xetex
              unicode=false, % unicode breaks when used with xetex
              xetex]{hyperref}
\else
  \usepackage[unicode=true]{hyperref}
\fi
\hypersetup{breaklinks=true,
            bookmarks=true,
            pdfauthor={},
            pdftitle={Rak piersi a czas przeżycia bez nawrotu choroby},
            colorlinks=true,
            citecolor=blue,
            urlcolor=blue,
            linkcolor=magenta,
            pdfborder={0 0 0}}
\urlstyle{same}  % don't use monospace font for urls
\setlength{\parindent}{0pt}
\setlength{\parskip}{6pt plus 2pt minus 1pt}
\setlength{\emergencystretch}{3em}  % prevent overfull lines
\setcounter{secnumdepth}{0}

%%% Use protect on footnotes to avoid problems with footnotes in titles
\let\rmarkdownfootnote\footnote%
\def\footnote{\protect\rmarkdownfootnote}

%%% Change title format to be more compact
\usepackage{titling}
\setlength{\droptitle}{-2em}
  \title{Rak piersi a czas przeżycia bez nawrotu choroby}
  \pretitle{\vspace{\droptitle}\centering\huge}
  \posttitle{\par}
  \author{}
  \preauthor{}\postauthor{}
  \date{}
  \predate{}\postdate{}


\usepackage{polski}
\usepackage[T1]{fontenc}
\usepackage[utf8]{inputenc} 
%\usepackage[top=1.5cm, bottom=1.5cm, left=0.85cm, right=0.85cm]{geometry}
\usepackage{fancyhdr}
\pagestyle{fancy}
\fancyhead[RO,LE]{\bfseries \small{P. Auguścik, M. Kosiński, B. Sozańska, A. Szewczyk}}
\fancyhead[RE,LO]{\bfseries \small{Biostatystyka, Projekt nr 1}}
\AtBeginDocument{\thispagestyle{fancy}}
\usepackage{rotating}
\usepackage{subfigure}
\usepackage{pdflscape}
\usepackage{amsfonts}
\usepackage{amsmath}
\usepackage{amssymb}
\usepackage{color}
\usepackage{amsthm}
\usepackage{longtable}
\usepackage{wrapfig,booktabs}
\usepackage{tikz}
\usepackage{float}
\usepackage{hyperref} %pakiet do dodawania hiperłącz
\hypersetup{colorlinks=true,
            linkcolor=black,
            citecolor=black,
            urlcolor=black}
%\title{\textbf{\LARGE{Biostatystyka - Projekt zaliczeniowy} }}


\begin{document}

\maketitle


\thispagestyle{fancy}

Pod analizę poddano 686 pacjentek cierpiących na raka piersi. Za cel
analizy postawiono pytanie, które zmienne mają wpływ na czas przeżycia
bez nawrotu choroby. Zaproponowano model
\text{proporcjonalnych hazardów (PH).}

\textbf{Sprawdzenie założeń - krzywe przeżycia i ich transformacje.}
\newline
Sprawdzono spełnianie założeń modelu dla każdej ze zmiennych (poza
ciągłymi) poprzez narysowanie krzywych Kaplana-Meiera (Rysunek 1) oraz
wykresów transformacji \text{\textsf{log(-log)}} krzywych przeżycia
\text{(Rysunek 2).}

\begin{figure}[hbt!]
  \vspace{-10pt}
  \begin{center}
   \subfigure[\textsf{meno}]{
     \includegraphics[width=0.18\textwidth, height=1.15in]{sc_meno.pdf}}
   \subfigure[\textsf{horm}]{
     \includegraphics[width=0.18\textwidth, height=1.15in]{sc_horm.pdf}}
   \subfigure[\textsf{prog}]{
     \includegraphics[width=0.18\textwidth, height=1.15in]{sc_prog.pdf}}
   \subfigure[\textsf{estr}]{
     \includegraphics[width=0.18\textwidth, height=1.15in]{sc_estr.pdf}}
   \subfigure[\textsf{grade}]{
     \includegraphics[width=0.18\textwidth, height=1.15in]{sc_grade.pdf}}
  \end{center}
  \vspace{-20pt}
  \label{fig:sc}
  \caption{Krzywe przeżycia Kaplana-Meiera dla zmiennych dyskretnych.}

\end{figure}

\begin{figure}[hbt!]
  \vspace{-10pt}
  \begin{center}
   \subfigure[\textsf{meno}]{
     \includegraphics[width=0.18\textwidth, height=1.15in]{sc_meno_log.pdf}}
   \subfigure[\textsf{horm}]{
     \includegraphics[width=0.18\textwidth, height=1.15in]{sc_horm_log.pdf}}
   \subfigure[\textsf{prog}]{
     \includegraphics[width=0.18\textwidth, height=1.15in]{sc_prog_log.pdf}}
   \subfigure[\textsf{estr}]{
     \includegraphics[width=0.18\textwidth, height=1.15in]{sc_estr_log.pdf}}
   \subfigure[\textsf{grade}]{
     \includegraphics[width=0.18\textwidth, height=1.15in]{sc_grade_log.pdf}}
  \end{center}
  \vspace{-20pt}
  \label{fig:sc}
  \caption{Transformacje \textsf{log(-log)} krzywych przeżycia Kaplana-Meiera.}

\end{figure}

Krzywe przeżycia dla różnych poziomów zmiennej \texttt{meno} przecinają
się, co oznacza niespełnianie założeń modelu PH. Dla pozostałych
zmiennych krzywe nie przecinają się oraz wykresy transformacji można
uznać za nieodstające od równoległych. Z tej racji zaproponowano model
proporcjonalnych hazardów warstwowany względem zmiennej \texttt{meno},
gdyż nie ma podstaw by nie twierdzić, że założenia modelu PH nie są
spełnione.

\begin{wraptable}{r}{6.8cm}
\vspace{-20pt}
\caption{ Wyniki testów logrank. }
\begin{tabular}{lrrrrrr}
\toprule%
\ &$\chi^2$&St. swobody&p-wartość\\ \toprule meno&0.3&1&0.597\\ horm&8.6&1&0.003\\ prog&49.3&1&0.000\\ estr&14.3&1&0.000\\ grade&44.5&2&0.000\\  \bottomrule
\end{tabular}
\vspace{-7.5pt}
\end{wraptable}

Aby potwierdzić wnioski płynące z graficznej prezentacji krzywych
przeżycia, przeprowadzono formalny test logrank dla każdej zmiennej,
którego podsumowanie widać w Tabeli 1. Dla zmiennej \textsf{grade}
przeprowadzono test logrank dla trendu z racji na naturalne
uporządkowanie poziomów. Wartości krytyczne dla zmiennych \textsf{horm},
\textsf{prog}, \textsf{estr} i \textsf{grade}(?), które są mniejsze
\text{od zakładanego} poziomu istotności \(\alpha=0.05\) dają
statystycznie istotne podstawy do odrzucenia hipotez zerowych w testach
logrank oraz do przyjęcia hipotez alternatywnych o tym, że krzywe
przeżycia dla danych zmiennych różnią się. Dla zmiennej \textsf{meno}
p-wartość \(0.597\) nie daje podstaw \text{do odrzucenia} hipotezy
zerowej (dla danego poziomu istnotności), mówiącej o równości krzywych
przeżycia dla różnych poziomów tej zmiennej.

\newpage
\textbf{Przekształcenia zmiennych ciągłych.} \newline
W celu sprawdzenia, czy zmienne ciągłe nie powinny zostać przekształcone
przed wprowadzeniem ich \text{do modelu}, wykonano wykresy reszt
martyngałowych pustego modelu od każdej z tych zmiennych. Wykresy
umieszczone na Rysunku 3 sugerują przekształcenie zmiennych poprzez
logarytm, więc dla potwierdzenia tych przypuszczeń narysowano jeszcze
wykresy reszt od logarytmów zmiennych i zaobserwowano, że wykresy są
teraz bliższe liniowym. Wprowadzono więc do modelu zmienne ciągłe
przekształcone logarytmem.

\begin{figure}[hbt!]
  \vspace{-10pt}
  \begin{center}
   \subfigure[\textsf{nodes}]{
     \includegraphics[width=0.32\textwidth, height=1.3in]{log_nodes.pdf}}
   \subfigure[\textsf{size}]{
     \includegraphics[width=0.32\textwidth, height=1.3in]{log_size.pdf}}
  \end{center}
  \vspace{-20pt}
  \label{fig:sc}
  \caption{Wykresy reszt martyngałowych od logarytmów zmiennych ciągłych.}

\end{figure}

\begin{wraptable}{r}{6.5cm}
\vspace{-12pt}
\caption{ Wyniki testu Schoenfelda. }
\begin{tabular}{lrrr}
\toprule%
  & $\rho$ & $\chi^2$ & p-wartość\\ \toprule 

horm & -0.009 & 0.024 & 0.876\\

prog & 0.036 & 0.369 & 0.543\\

estr & 0.071 & 1.458 & 0.227\\

as.factor(grade)2 & -0.052 & 0.811 & 0.368\\

as.factor(grade)3 & -0.089 & 2.347 & 0.126\\

log(size) & 0.011 & 0.037 & 0.848\\

log(nodes) & -0.051 & 0.879 & 0.349\\

GLOBAL & NA & 11.322 & 0.125\\  \bottomrule
\end{tabular}
\vspace{-7.5pt}
\end{wraptable}

\textbf{Sprawdzenie założeń - formalny test Schoenfelda.} \newline
W celu formalnego sprawdzenia czy współczynniki w modelu są stałe w
czasie przeprowadzono test Schoenfelda, którego globalna p-wartość oraz
pojedyncze p-wartości dla zmiennych dyskretnych oraz ciągłych
przekształconych przez logarytm są większe \text{od zakładanego} poziomu
istotności \(\alpha=0.05\) co nie daje podstaw \text{do odrzucenia}
hipotezy zerowej w tym teście, mówiącej \text{o stałości} współczynników
w czasie.

Przyjrzano się również wykresom skalowanych reszt Schoenfelda dla każdej
zmiennej w modelu od czasu i dopasowano krzywe do
\textcolor{red}{danych}?. Dołączone granice dla 95\% obszarów ufności
nie dają podstawy do odrzucenia hipotez, że krzywe nie różnią się od
horyzontalnych.

\begin{figure}[hbt!]
\vspace{-10pt}
  \begin{center}
      \includegraphics[width=0.85\textwidth, height=3in]{skal_res_shen.pdf}
      \caption{Wykresy skalowanych reszt Schoenfelda.}
   \end{center}
\end{figure}

Zarówno wykresy skalowanych reszt Schoenfelda, jak i wyniki testu nie
sugerują odstępstw od założenia PH.

\textbf{Dopasowanie modelu proporcjonalnych hazardów.} \newline

\begin{Shaded}
\begin{Highlighting}[]
\NormalTok{model <-}\StringTok{ }\KeywordTok{coxph}\NormalTok{(}\KeywordTok{Surv}\NormalTok{(rectime,censrec)~horm+prog+estr+}\KeywordTok{as.factor}\NormalTok{(grade)+}\KeywordTok{strata}\NormalTok{(meno)+}\KeywordTok{log}\NormalTok{(size)+}\KeywordTok{log}\NormalTok{(nodes), }
               \DataTypeTok{data =} \NormalTok{dane) }
\KeywordTok{summary}\NormalTok{(model)}
\end{Highlighting}
\end{Shaded}

\begin{verbatim}
Call:
coxph(formula = Surv(rectime, censrec) ~ horm + prog + estr + 
    as.factor(grade) + strata(meno) + log(size) + log(nodes), 
    data = dane)

  n= 686, number of events= 299 

                      coef exp(coef) se(coef)      z Pr(>|z|)    
horm              -0.40288   0.66839  0.12924 -3.117  0.00183 ** 
prog              -0.67857   0.50734  0.14434 -4.701 2.59e-06 ***
estr               0.03400   1.03458  0.14314  0.238  0.81226    
as.factor(grade)2  0.50378   1.65497  0.25275  1.993  0.04624 *  
as.factor(grade)3  0.52144   1.68445  0.27898  1.869  0.06161 .  
log(size)          0.20252   1.22449  0.13207  1.533  0.12515    
log(nodes)         0.47991   1.61593  0.06697  7.166 7.71e-13 ***
---
Signif. codes:  0 '***' 0.001 '**' 0.01 '*' 0.05 '.' 0.1 ' ' 1

                  exp(coef) exp(-coef) lower .95 upper .95
horm                 0.6684     1.4961    0.5188    0.8611
prog                 0.5073     1.9711    0.3823    0.6732
estr                 1.0346     0.9666    0.7815    1.3696
as.factor(grade)2    1.6550     0.6042    1.0084    2.7160
as.factor(grade)3    1.6845     0.5937    0.9750    2.9102
log(size)            1.2245     0.8167    0.9452    1.5862
log(nodes)           1.6159     0.6188    1.4172    1.8426

Concordance= 0.697  (se = 0.025 )
Rsquare= 0.172   (max possible= 0.99 )
Likelihood ratio test= 129.7  on 7 df,   p=0
Wald test            = 128.6  on 7 df,   p=0
Score (logrank) test = 136.5  on 7 df,   p=0
\end{verbatim}

W stworzonym modelu zmiennymi istotnymi statystycznie na poziomie
istotności \(0.05\) są: \texttt{horm}, \texttt{prog},
\texttt{as.factor(grade)2}, log(\texttt{nodes}). Współczynniki modelu
przy tych zmiennych wynoszą odpowiednio: \(-0.40, -0.67, 0.50, 0.47\).
Oznacza to, że jeśli została użyta terapia hormonalna, to hazard zgonu
lub nawrotu choroby zmienia się \(\exp(-0.40)=0.66\) raza w stosunku do
nieużycia terapii hormonalnej (gdy wszystkie inne zmienne są takie
same). Jeśli wskaźnik receptorów progesteronu zmienia się z ujemnego na
dodatni, to hazard zmieni się o \(\exp(-0.67)=0.5\) raza, natomiast gdy
logarytm liczby węzłów chłonnych z przerzutami nowotworu wzrośnie o 1,
to hazard zwiększy się \(\exp(0.47)=1.61\) raza. Zmiana z grade (stopień
zróżnicowania komórek nowotworu (1--wysoki, 2--średni) z 1 do 2 zwiększa
hazard \(\exp(0.5)=1.65\) raza.

\textcolor{red}{Testy Likelihood, Wald i Score dają p-wartość mniejszą od 0.05, co oznacza, że odrzucamy hipotezę o tym, że model pusty jest równie dobrze dopasowany jak dopasowany model na korzyść hipotezy, że dopasowany model jest od niego lepszy. <- poprawić.}

\newpage
\textbf{Sprawdzenie dopasowania modelu proporcjonalnych hazardów.}
\newline
W celu sprawdzenia dopasowania modelu wygenerowano wykresy reszt
dewiancji/martyngałowych od liniowej kombinacji zmiennych/indeksów. Na
niektórych wykresach można wskazać wyraźnie odstające reszty. Wykres
reszt dewiancji od indeksów można uznać za symetryczny, pozostałe
niekoniecznie, \text{co sugeruje}, że dopasowany model nie jest
perfekcyjny.

\begin{figure}[hbt!]
  \vspace{-10pt}
  \begin{center}
   \subfigure[Reszty dewiancji a indeks.]{
     \includegraphics[width=0.48\textwidth, height=2in]{res_ind_dev.pdf}}
   \subfigure[Reszty martyngałowe a indeks.]{
     \includegraphics[width=0.48\textwidth, height=2in]{res_ind_mart.pdf}}
   \subfigure[Reszty dewiancji a liniowe kombinacje.]{
     \includegraphics[width=0.48\textwidth, height=2in]{res_lp_dev.pdf}}
   \subfigure[Reszty martyngałowe a liniowe kombinacje.]{
     \includegraphics[width=0.48\textwidth, height=2in]{res_lp_mart.pdf}}
  \end{center}
  \vspace{-20pt}
  \label{fig:sc}
  \caption{Wykresy reszt martyngałowych i dewiancji.}

\end{figure}

\newpage
\textbf{Kody.} \newline

\begin{Shaded}
\begin{Highlighting}[]
\KeywordTok{library}\NormalTok{(foreign)}
\NormalTok{dane <-}\StringTok{ }\KeywordTok{read.dta}\NormalTok{(}\StringTok{"gbcs_short.dta"}\NormalTok{)}
\KeywordTok{library}\NormalTok{(survival)}
\KeywordTok{plot}\NormalTok{(}\KeywordTok{survfit}\NormalTok{(}\KeywordTok{Surv}\NormalTok{(rectime,censrec)~meno, }\DataTypeTok{data =} \NormalTok{dane), }\DataTypeTok{col=}\KeywordTok{c}\NormalTok{(}\StringTok{"orange"}\NormalTok{,}\StringTok{"purple"}\NormalTok{), }\DataTypeTok{lty=}\KeywordTok{c}\NormalTok{(}\DecValTok{1}\NormalTok{:}\DecValTok{2}\NormalTok{), }\DataTypeTok{lwd=}\DecValTok{3}\NormalTok{)}
\CommentTok{#...}
\KeywordTok{plot}\NormalTok{(}\KeywordTok{log}\NormalTok{(dane$size), }\KeywordTok{resid}\NormalTok{(}\KeywordTok{coxph}\NormalTok{(}\KeywordTok{Surv}\NormalTok{(rectime,censrec)~}\DecValTok{1}\NormalTok{, }\DataTypeTok{data =} \NormalTok{dane)))}
\KeywordTok{lines}\NormalTok{(}\KeywordTok{lowess}\NormalTok{(}\KeywordTok{log}\NormalTok{(dane$size), }\KeywordTok{resid}\NormalTok{(}\KeywordTok{coxph}\NormalTok{(}\KeywordTok{Surv}\NormalTok{(rectime,censrec)~}\DecValTok{1}\NormalTok{, }\DataTypeTok{data =} \NormalTok{dane)), }\DataTypeTok{iter =} \DecValTok{0}\NormalTok{, }\DataTypeTok{f =} \FloatTok{0.6}\NormalTok{))}
\CommentTok{#...}

\KeywordTok{par}\NormalTok{(}\DataTypeTok{mfrow=}\KeywordTok{c}\NormalTok{(}\DecValTok{2}\NormalTok{,}\DecValTok{3}\NormalTok{))}
\KeywordTok{plot}\NormalTok{(}\KeywordTok{cox.zph}\NormalTok{(model, }\DataTypeTok{transform =} \StringTok{"identity"}\NormalTok{), }\DataTypeTok{df=}\DecValTok{4}\NormalTok{,}\DataTypeTok{nsmo=}\DecValTok{10}\NormalTok{, }\DataTypeTok{se=}\OtherTok{TRUE}\NormalTok{)}
\NormalTok{model <-}\StringTok{ }\KeywordTok{coxph}\NormalTok{(}\KeywordTok{Surv}\NormalTok{(rectime,censrec)~horm+prog+estr+grade+}\KeywordTok{strata}\NormalTok{(meno)+}\KeywordTok{log}\NormalTok{(size)+}\KeywordTok{log}\NormalTok{(nodes), }
               \DataTypeTok{data =} \NormalTok{dane) }
\KeywordTok{summary}\NormalTok{(model)}

\CommentTok{#...}
\KeywordTok{qplot}\NormalTok{(}\DecValTok{1}\NormalTok{:}\DecValTok{686}\NormalTok{,}\KeywordTok{residuals}\NormalTok{(model, }\DataTypeTok{type=}\StringTok{"deviance"}\NormalTok{))+}\KeywordTok{xlab}\NormalTok{(}\StringTok{"index"}\NormalTok{)+}\KeywordTok{theme_tufte}\NormalTok{(}\DataTypeTok{base_size=}\DecValTok{20}\NormalTok{)+}\KeywordTok{geom_hline}\NormalTok{()+}
\StringTok{   }\KeywordTok{geom_hline}\NormalTok{(}\DataTypeTok{yintercept=}\DecValTok{0}\NormalTok{, }\DataTypeTok{col =}\StringTok{"orange"}\NormalTok{, }\DataTypeTok{size =} \DecValTok{3}\NormalTok{)}
\end{Highlighting}
\end{Shaded}

\end{document}
