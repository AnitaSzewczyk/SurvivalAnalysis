\documentclass[]{article}
\usepackage{lmodern}
\usepackage{amssymb,amsmath}
\usepackage{ifxetex,ifluatex}
\usepackage{fixltx2e} % provides \textsubscript
\ifnum 0\ifxetex 1\fi\ifluatex 1\fi=0 % if pdftex
  \usepackage[T1]{fontenc}
  \usepackage[utf8]{inputenc}
\else % if luatex or xelatex
  \ifxetex
    \usepackage{mathspec}
    \usepackage{xltxtra,xunicode}
  \else
    \usepackage{fontspec}
  \fi
  \defaultfontfeatures{Mapping=tex-text,Scale=MatchLowercase}
  \newcommand{\euro}{€}
\fi
% use upquote if available, for straight quotes in verbatim environments
\IfFileExists{upquote.sty}{\usepackage{upquote}}{}
% use microtype if available
\IfFileExists{microtype.sty}{%
\usepackage{microtype}
\UseMicrotypeSet[protrusion]{basicmath} % disable protrusion for tt fonts
}{}
\usepackage[margin=1in]{geometry}
\usepackage{color}
\usepackage{fancyvrb}
\newcommand{\VerbBar}{|}
\newcommand{\VERB}{\Verb[commandchars=\\\{\}]}
\DefineVerbatimEnvironment{Highlighting}{Verbatim}{commandchars=\\\{\}}
% Add ',fontsize=\small' for more characters per line
\newenvironment{Shaded}{}{}
\newcommand{\KeywordTok}[1]{\textcolor[rgb]{0.00,0.44,0.13}{\textbf{{#1}}}}
\newcommand{\DataTypeTok}[1]{\textcolor[rgb]{0.56,0.13,0.00}{{#1}}}
\newcommand{\DecValTok}[1]{\textcolor[rgb]{0.25,0.63,0.44}{{#1}}}
\newcommand{\BaseNTok}[1]{\textcolor[rgb]{0.25,0.63,0.44}{{#1}}}
\newcommand{\FloatTok}[1]{\textcolor[rgb]{0.25,0.63,0.44}{{#1}}}
\newcommand{\CharTok}[1]{\textcolor[rgb]{0.25,0.44,0.63}{{#1}}}
\newcommand{\StringTok}[1]{\textcolor[rgb]{0.25,0.44,0.63}{{#1}}}
\newcommand{\CommentTok}[1]{\textcolor[rgb]{0.38,0.63,0.69}{\textit{{#1}}}}
\newcommand{\OtherTok}[1]{\textcolor[rgb]{0.00,0.44,0.13}{{#1}}}
\newcommand{\AlertTok}[1]{\textcolor[rgb]{1.00,0.00,0.00}{\textbf{{#1}}}}
\newcommand{\FunctionTok}[1]{\textcolor[rgb]{0.02,0.16,0.49}{{#1}}}
\newcommand{\RegionMarkerTok}[1]{{#1}}
\newcommand{\ErrorTok}[1]{\textcolor[rgb]{1.00,0.00,0.00}{\textbf{{#1}}}}
\newcommand{\NormalTok}[1]{{#1}}
\ifxetex
  \usepackage[setpagesize=false, % page size defined by xetex
              unicode=false, % unicode breaks when used with xetex
              xetex]{hyperref}
\else
  \usepackage[unicode=true]{hyperref}
\fi
\hypersetup{breaklinks=true,
            bookmarks=true,
            pdfauthor={},
            pdftitle={Czas przezycia bez nawrotu choroby -- parametryczna postac modelu},
            colorlinks=true,
            citecolor=blue,
            urlcolor=blue,
            linkcolor=magenta,
            pdfborder={0 0 0}}
\urlstyle{same}  % don't use monospace font for urls
\setlength{\parindent}{0pt}
\setlength{\parskip}{6pt plus 2pt minus 1pt}
\setlength{\emergencystretch}{3em}  % prevent overfull lines
\setcounter{secnumdepth}{0}

%%% Use protect on footnotes to avoid problems with footnotes in titles
\let\rmarkdownfootnote\footnote%
\def\footnote{\protect\rmarkdownfootnote}

%%% Change title format to be more compact
\usepackage{titling}
\setlength{\droptitle}{-2em}
  \title{Czas przezycia bez nawrotu choroby -- parametryczna postac modelu}
  \pretitle{\vspace{\droptitle}\centering\huge}
  \posttitle{\par}
  \author{}
  \preauthor{}\postauthor{}
  \date{}
  \predate{}\postdate{}


\usepackage{polski}
\usepackage[T1]{fontenc}
\usepackage[utf8]{inputenc} 
%\usepackage[top=1.5cm, bottom=1.5cm, left=0.85cm, right=0.85cm]{geometry}
\usepackage{fancyhdr}
\pagestyle{fancy}
\fancyhead[RO,LE]{\bfseries \small{P. Auguścik, M. Kosiński, B. Sozańska, A. Szewczyk}}
\fancyhead[RE,LO]{\bfseries \small{Biostatystyka, Projekt nr 2}}
\AtBeginDocument{\thispagestyle{fancy}}
\usepackage{rotating}
\usepackage{subfigure}
\usepackage{pdflscape}

\usepackage{amsfonts}
\usepackage{amsmath}
\usepackage{amssymb}
\usepackage{color}
\usepackage{amsthm}
\usepackage{longtable}
\usepackage{wrapfig,booktabs}
\usepackage{tikz}
\usepackage{float}
\usepackage{hyperref} %pakiet do dodawania hiperłącz
\hypersetup{colorlinks=true,
            linkcolor=black,
            citecolor=black,
            urlcolor=black}
%\title{\textbf{\LARGE{Biostatystyka - Projekt zaliczeniowy bardzo fajny} }}


\begin{document}

\maketitle


\thispagestyle{fancy}

Chcac sprawdzic, jakie czynniki maja istotnie statystyczny wplyw na czas
do zdarzenia jakim jest zgon przy analizie danych dotyczacych raka
piersi, zaproponowano sprawdzenie czy modele parametryczne zakladajace
postac rozkladu czasu do zdarzenia sa adekwatne w danym problemie
medycznym.

\textbf{Wybór parametrycznej formy modelu.}

Sprawdzajac postac parametryczna dla modelu AFT postanowiono wybrac
rozklad z bogatej rodziny uogólnionych rozkladów F. Korzystajac z
pakietu \texttt{flexsurv} dopasowano modele Weibulla, Log-logistyczny,
\text{Log-normalny}, Uogólniony Gamma oraz Uogólniony F.

\begin{Shaded}
\begin{Highlighting}[]
\NormalTok{AFT.GG <-}\StringTok{ }\KeywordTok{flexsurvreg}\NormalTok{(}\KeywordTok{Surv}\NormalTok{(rectime,censrec)~horm+prog+estr+}\KeywordTok{as.factor}\NormalTok{(grade)+meno+size+nodes, }
                             \DataTypeTok{data =} \NormalTok{dane, }\DataTypeTok{dist=}\StringTok{"gengamma"}\NormalTok{)}
\NormalTok{AFT.GF <-}\StringTok{ }\KeywordTok{flexsurvreg}\NormalTok{(}\KeywordTok{Surv}\NormalTok{(rectime,censrec)~horm+prog+estr+}\KeywordTok{as.factor}\NormalTok{(grade)+meno+size+nodes, }
                             \DataTypeTok{data =} \NormalTok{dane, }\DataTypeTok{dist=}\StringTok{"genf"}\NormalTok{)}
\NormalTok{AFT.LL <-}\StringTok{ }\KeywordTok{flexsurvreg}\NormalTok{(}\KeywordTok{Surv}\NormalTok{(rectime,censrec)~horm+prog+estr+}\KeywordTok{as.factor}\NormalTok{(grade)+meno+size+nodes, }
                             \DataTypeTok{data =} \NormalTok{dane, }\DataTypeTok{dist=}\StringTok{"genf"}\NormalTok{, }\DataTypeTok{inits=}\KeywordTok{c}\NormalTok{(}\DecValTok{3}\NormalTok{,}\FloatTok{0.2}\NormalTok{,}\DecValTok{0}\NormalTok{,}\DecValTok{1}\NormalTok{,}\DecValTok{0}\NormalTok{,}\DecValTok{0}\NormalTok{,}\DecValTok{0}\NormalTok{,}\DecValTok{0}\NormalTok{,}\DecValTok{0}\NormalTok{,}\DecValTok{0}\NormalTok{,}\DecValTok{0}\NormalTok{,}\DecValTok{0}\NormalTok{), }
                             \DataTypeTok{fixedpars =} \KeywordTok{c}\NormalTok{(}\DecValTok{3}\NormalTok{,}\DecValTok{4}\NormalTok{))}
\NormalTok{AFT.Weibull <-}\StringTok{ }\KeywordTok{flexsurvreg}\NormalTok{(}\KeywordTok{Surv}\NormalTok{(rectime,censrec)~horm+prog+estr+}\KeywordTok{as.factor}\NormalTok{(grade)+meno+}
\StringTok{                             }\NormalTok{size+nodes,}
                             \DataTypeTok{data =} \NormalTok{dane, }\DataTypeTok{dist=}\StringTok{"weibull"}\NormalTok{)}
\NormalTok{AFT.LN <-}\StringTok{ }\KeywordTok{flexsurvreg}\NormalTok{(}\KeywordTok{Surv}\NormalTok{(rectime,censrec)~horm+prog+estr+}\KeywordTok{as.factor}\NormalTok{(grade)+meno+size+nodes, }
                             \DataTypeTok{data =} \NormalTok{dane, }\DataTypeTok{dist=}\StringTok{"lnorm"}\NormalTok{)}
\end{Highlighting}
\end{Shaded}

W celu oceny, który model jest adekwatny, przeprowdzono testy ilorazu
wiarogodnosci, jak ponizej.

Wartosci logarytmów funkcji wiarogodnosci dla modeli:

\begin{table}[hbt!]
\centering
\begin{tabular}{lr}
\toprule%
  & loglik\\ \toprule%

Gen Gamma & -2553.051\\

Gen F & -2553.067\\

Log-logistic & -2563.449\\

Weibull & -2575.998\\

Log-normal & -2555.798\\
\bottomrule
\end{tabular}
\caption{Wartosci logarytmów funkcji wiarogodnosci dla modeli parametrycznych.}
\end{table}

P-wartosci testów:

\begin{table}[hbt!]
\centering
\begin{tabular}{lr}
\toprule%
  & p-wartosc\\
\toprule%
GF vs GG & 1.0000\\

GG vs LL & 0.0000\\

GG vs Wei & 0.0000\\

GG vs LN & 0.0641\\
\bottomrule
\end{tabular}
\caption{Wartosci krytycze testów.}
\end{table}

Testujac mozliowsc uzycia danego rozkladu, przeprowadzono testy dla
uogólnionego rozladu gamma \text{i uogólnionego} F. Stwierdzono, na
poziomie istotnosci \(\alpha=0.05\), po poprawce Bonferroniego
uwzgledniajacej 4 testy, czyli na poziomie istotnosci dla pojedycznego
testu równym \(\alpha_i = 0.0125, i =1,2,3,4\), ze nie ma podstaw do
odrzucenia hipotezy zerowej w tescie sprawdzajacym czy model
\text{z mniejszej} rodziny uogólnionych rozkladów gamma jest wlasciwy w
porównaniu do wiekszego modelu z rodziny uogólnionych rozkladów F.
Nastepnie dla rozkladu z uogólnionej rodziny rozkladów gamma
przeprowadzono 3 testy sprawdzajace, czy modele z mniejszej rodziny
(log-normalny, log-logistyczny, Weibulla) sa wlasciwe
\text{w porównaniu} do rozkladu z uogólnionej rodziny rozkladów gamma.
Tylko w przypadku rozkladu log-normalnego nie ma podstaw
\text{do odrzucenia} hipotezy, ze ten rozklad jest wlasciwy w porównaniu
\text{do rozkladu} z szerszej rodziny rozkladów uogólnionych gamma. Dla
rozkladu log-logistycznego i Weibulla odrzucono hipoteze zerowa o tym,
ze te rozklady sa wlasciwe w stosunku do rozkladu z szerszej rodziny
rozkladów uogólnionych gamma.

Zatem w dalszej czesci raportu sprawdzamy dopasowanie modelu
log-normalnego.

\textbf{Sprawdzenie dopasowania modelu log-normalnego.}

\begin{figure}[hbt!]
  \vspace{-20pt}
  \begin{center}
     \includegraphics[width=0.7\textwidth, height=3in]{lognormal.pdf}
  \end{center}
  \vspace{-20pt}
  \label{fig:sc}
  \caption{Reszty modelu a cenzurowana próbka z rozkladu log-normalnego.}

\end{figure}

Sprawdzenie, czy reszty modelu zachowuja sie jak cenzurowana próbka z
rozkladu log-normalnego, pokazane jest na Rysunku 1. Widac z wykresu, ze
nie ma widocznych odstepstw miedzy krzywymi, co moze swiadczyc o dobrym
dopasowaniu modelu. Parametryczne zalozenie, ze reszty pochodza z
rozkladu log-normalnego wydaje sie byc spelnione na podstawie
\text{Rysunku 1.}

\textbf{ Podsumownie modelu log-normalnego.}

Podsumowanie modelu mozna uzyskac poleceniem jak ponizej:

\begin{Shaded}
\begin{Highlighting}[]
\KeywordTok{psm}\NormalTok{(}\KeywordTok{Surv}\NormalTok{(rectime,censrec)~horm+prog+estr+}\KeywordTok{as.factor}\NormalTok{(grade)+meno+size+nodes, }
        \DataTypeTok{data =} \NormalTok{dane, }\DataTypeTok{dist =} \StringTok{"lognormal"}\NormalTok{)}
\end{Highlighting}
\end{Shaded}

\begin{verbatim}

Parametric Survival Model: Log Normal Distribution

psm(formula = Surv(rectime, censrec) ~ horm + prog + estr + as.factor(grade) + 
    meno + size + nodes, data = dane, dist = "lognormal")

                    Model Likelihood     Discrimination    
                       Ratio Test           Indexes        
Obs         686    LR chi2     126.17    R2       0.168    
Events      299    d.f.             8    Dxy      0.382    
sigma 0.9690015    Pr(> chi2) <0.0001    g        0.036    
                                         gr       0.564    

            Coef    S.E.   Wald Z Pr(>|Z|)
(Intercept)  7.4967 0.2670 28.08  <0.0001 
horm         0.3339 0.0958  3.49  0.0005  
prog         0.4995 0.1097  4.55  <0.0001 
estr         0.0418 0.1083  0.39  0.6992  
grade=2     -0.4396 0.1643 -2.68  0.0075  
grade=3     -0.4780 0.1880 -2.54  0.0110  
meno        -0.0427 0.0932 -0.46  0.6469  
size        -0.0056 0.0031 -1.82  0.0685  
nodes       -0.0483 0.0080 -6.08  <0.0001 
Log(scale)  -0.0315 0.0443 -0.71  0.4773  
\end{verbatim}

Zmiennymi istotnymi w modelu sa \texttt{horm}, \texttt{prog}, oba
poziomy zmiennej \texttt{grade} wzgledem poziomu referencyjnego oraz
\texttt{nodes}. Wspólczynniki modelu dla tych zmiennych wynosza
odpowiednio 0.3339, 0.4995, -0.4396, -0.4780, -0.0483. Oznacza to, ze
czas do nawrotu choroby przy uzyciu terapii hormonalnej jest dluzszy
\(e^{0.3339} = 1.4\) raza w porównaniu do sytuacji gdy nie jest
stosowana terapia hormonalna, przy zalozeniu stalosci pozostalych
zmiennych. Czas do nawrotu choroby przy dodatnim wskazniku progesteronu
jest dluzszy \(e^{0.4995} = 1.6\) raza w porównaniu do sytuacji gdy
pacjentka posiada ujemny wskaznik poziomu progesteronu, przy zalozeniu
stalosci pozostalych zmiennych. Czas do nawrotu choroby przy srednim
stopniu zróznicowania komórek nowotworu jest krótszy
\(e^{-0.4396} = 0.64\) raza w porównaniu do sytuacji gdy pacjentka
posiada wysoki stopien zróznicowania komórek nowotworu, przy zalozeniu
stalosci pozostalych zmiennych. Natomiast gdy pacjentka posiada niski
stopien zróznicowania nowotworu, to czas do nawrotu choroby jest krótszy
\(e^{-0.4780} = 0.62\) raza w porównaniu do sytuacji gdy pacjentka
posiada wysoki stopien zróznicowania komórek nowotworu, przy zalozeniu
stalosci pozostalych zmiennych. Czas do nawrotu choroby przy wzroscie
liczby wezlów chlonnych \text{z przerzutami} nowotworu o jeden jest
krótszy \(e^{-0.0483} = 0.95\) raza, przy zalozeniu stalosci pozostalych
zmiennych.

\newpage
\textbf{Kody:}

\begin{Shaded}
\begin{Highlighting}[]
\KeywordTok{library}\NormalTok{(foreign)}
\NormalTok{dane <-}\StringTok{ }\KeywordTok{read.dta}\NormalTok{(}\StringTok{"gbcs_short.dta"}\NormalTok{)}
\KeywordTok{library}\NormalTok{(survival)}
\KeywordTok{library}\NormalTok{(flexsurv)}
\KeywordTok{library}\NormalTok{(rms)}

\CommentTok{#modele:}
\NormalTok{AFT.GG <-}\StringTok{ }\KeywordTok{flexsurvreg}\NormalTok{(}\KeywordTok{Surv}\NormalTok{(rectime,censrec)~horm+prog+estr+}\KeywordTok{as.factor}\NormalTok{(grade)+meno+size+nodes, }
                             \DataTypeTok{data =} \NormalTok{dane, }\DataTypeTok{dist=}\StringTok{"gengamma"}\NormalTok{)}
\NormalTok{AFT.GF <-}\StringTok{ }\KeywordTok{flexsurvreg}\NormalTok{(}\KeywordTok{Surv}\NormalTok{(rectime,censrec)~horm+prog+estr+}\KeywordTok{as.factor}\NormalTok{(grade)+meno+size+nodes, }
                             \DataTypeTok{data =} \NormalTok{dane, }\DataTypeTok{dist=}\StringTok{"genf"}\NormalTok{)}
\NormalTok{AFT.LL <-}\StringTok{ }\KeywordTok{flexsurvreg}\NormalTok{(}\KeywordTok{Surv}\NormalTok{(rectime,censrec)~horm+prog+estr+}\KeywordTok{as.factor}\NormalTok{(grade)+meno+size+nodes, }
                             \DataTypeTok{data =} \NormalTok{dane, }\DataTypeTok{dist=}\StringTok{"genf"}\NormalTok{, }\DataTypeTok{inits=}\KeywordTok{c}\NormalTok{(}\DecValTok{3}\NormalTok{,}\FloatTok{0.2}\NormalTok{,}\DecValTok{0}\NormalTok{,}\DecValTok{1}\NormalTok{,}\DecValTok{0}\NormalTok{,}\DecValTok{0}\NormalTok{,}\DecValTok{0}\NormalTok{,}\DecValTok{0}\NormalTok{,}\DecValTok{0}\NormalTok{,}\DecValTok{0}\NormalTok{,}\DecValTok{0}\NormalTok{,}\DecValTok{0}\NormalTok{), }
                      \DataTypeTok{fixedpars =} \KeywordTok{c}\NormalTok{(}\DecValTok{3}\NormalTok{,}\DecValTok{4}\NormalTok{))}
\NormalTok{AFT.Weibull <-}\StringTok{ }\KeywordTok{flexsurvreg}\NormalTok{(}\KeywordTok{Surv}\NormalTok{(rectime,censrec)~horm+prog+estr+}\KeywordTok{as.factor}\NormalTok{(grade)+meno+size+nodes, }
                              \DataTypeTok{data =} \NormalTok{dane, }\DataTypeTok{dist=}\StringTok{"weibull"}\NormalTok{)}
\NormalTok{AFT.LN <-}\StringTok{ }\KeywordTok{flexsurvreg}\NormalTok{(}\KeywordTok{Surv}\NormalTok{(rectime,censrec)~horm+prog+estr+}\KeywordTok{as.factor}\NormalTok{(grade)+meno+size+nodes, }
                              \DataTypeTok{data =} \NormalTok{dane, }\DataTypeTok{dist=}\StringTok{"lnorm"}\NormalTok{)}

\CommentTok{#Wartosci logarytmów funkcji wiarogodnosci dla modeli:}
\KeywordTok{matrix}\NormalTok{( }\KeywordTok{c}\NormalTok{(AFT.GG$loglik, }\CommentTok{# loglik generalized G }
\NormalTok{AFT.GF$loglik, }\CommentTok{# loglik generalized F}
\NormalTok{AFT.LL$loglik, }\CommentTok{# logik log-logistic}
\NormalTok{AFT.Weibull$loglik, }\CommentTok{# logik Weibull}
\NormalTok{AFT.LN$loglik), }\DataTypeTok{ncol=}\DecValTok{1}\NormalTok{) ->x50 }\CommentTok{# logik log-normalny}
\KeywordTok{rownames}\NormalTok{(x50) <-}\StringTok{ }\KeywordTok{c}\NormalTok{(}\StringTok{"Gen Gamma"}\NormalTok{, }\StringTok{"Gen F"}\NormalTok{, }\StringTok{"Log-logistic"}\NormalTok{, }\StringTok{"Weibull"}\NormalTok{, }\StringTok{"Log-normal"}\NormalTok{)}
\KeywordTok{colnames}\NormalTok{(x50) <-}\StringTok{ "loglik"}

\CommentTok{#p-wartosci testóW:}
\KeywordTok{matrix}\NormalTok{(}\KeywordTok{round}\NormalTok{(}\KeywordTok{c}\NormalTok{(}\DecValTok{1}\NormalTok{-}\KeywordTok{pchisq}\NormalTok{(}\DecValTok{2}\NormalTok{*(AFT.GF$loglik-AFT.GG$loglik),}\DecValTok{2}\NormalTok{),}
\DecValTok{1}\NormalTok{-}\KeywordTok{pchisq}\NormalTok{(}\DecValTok{2}\NormalTok{*(AFT.GG$loglik-AFT.LL$loglik),}\DecValTok{2}\NormalTok{),}
\DecValTok{1}\NormalTok{-}\KeywordTok{pchisq}\NormalTok{(}\DecValTok{2}\NormalTok{*(AFT.GG$loglik-AFT.Weibull$loglik),}\DecValTok{2}\NormalTok{), }
\DecValTok{1}\NormalTok{-}\KeywordTok{pchisq}\NormalTok{(}\DecValTok{2}\NormalTok{*(AFT.GG$loglik-AFT.LN$loglik),}\DecValTok{2}\NormalTok{)), }\DataTypeTok{digits=}\DecValTok{4}\NormalTok{), }\DataTypeTok{ncol=}\DecValTok{1}\NormalTok{) ->}\StringTok{ }\NormalTok{x65}
\KeywordTok{rownames}\NormalTok{(x65) <-}\StringTok{ }\KeywordTok{c}\NormalTok{(}\StringTok{"GF vs GG"}\NormalTok{, }\StringTok{"GG vs LL"}\NormalTok{, }\StringTok{"GG vs Wei"}\NormalTok{, }\StringTok{"GG vs LN"}\NormalTok{)}
\KeywordTok{colnames}\NormalTok{(x65) <-}\StringTok{ "p-wartosc"}

\CommentTok{#Sprawdzenie dopasowania modelu log-normalnego}
\NormalTok{logNorx1 <-}\StringTok{ }\KeywordTok{psm}\NormalTok{(}\KeywordTok{Surv}\NormalTok{(rectime,censrec)~horm+prog+estr+}\KeywordTok{as.factor}\NormalTok{(grade)+meno+size+nodes, }
        \DataTypeTok{data =} \NormalTok{dane, }\DataTypeTok{dist =} \StringTok{"lognormal"}\NormalTok{) }
\NormalTok{res.LogN1 <-}\StringTok{ }\KeywordTok{resid}\NormalTok{(logNorx1,}\DataTypeTok{type=}\StringTok{"cens"}\NormalTok{)}

\KeywordTok{survplot}\NormalTok{(}\KeywordTok{npsurv}\NormalTok{(res.LogN1 ~}\DecValTok{1}\NormalTok{),}\DataTypeTok{conf=}\StringTok{"none"}\NormalTok{,}\DataTypeTok{ylab=}\StringTok{"Survival probability"}\NormalTok{, }\DataTypeTok{xlab=}\StringTok{"Residual"}\NormalTok{)}
\KeywordTok{lines}\NormalTok{(res.LogN1)}
\KeywordTok{dev.off}\NormalTok{()}
\end{Highlighting}
\end{Shaded}

\end{document}
