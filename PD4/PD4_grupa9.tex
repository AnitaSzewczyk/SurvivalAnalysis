\documentclass[]{article}
\usepackage{lmodern}
\usepackage{amssymb,amsmath}
\usepackage{ifxetex,ifluatex}
\usepackage{fixltx2e} % provides \textsubscript
\ifnum 0\ifxetex 1\fi\ifluatex 1\fi=0 % if pdftex
  \usepackage[T1]{fontenc}
  \usepackage[utf8]{inputenc}
\else % if luatex or xelatex
  \ifxetex
    \usepackage{mathspec}
    \usepackage{xltxtra,xunicode}
  \else
    \usepackage{fontspec}
  \fi
  \defaultfontfeatures{Mapping=tex-text,Scale=MatchLowercase}
  \newcommand{\euro}{€}
\fi
% use upquote if available, for straight quotes in verbatim environments
\IfFileExists{upquote.sty}{\usepackage{upquote}}{}
% use microtype if available
\IfFileExists{microtype.sty}{%
\usepackage{microtype}
\UseMicrotypeSet[protrusion]{basicmath} % disable protrusion for tt fonts
}{}
\usepackage[margin=1in]{geometry}
\usepackage{color}
\usepackage{fancyvrb}
\newcommand{\VerbBar}{|}
\newcommand{\VERB}{\Verb[commandchars=\\\{\}]}
\DefineVerbatimEnvironment{Highlighting}{Verbatim}{commandchars=\\\{\}}
% Add ',fontsize=\small' for more characters per line
\newenvironment{Shaded}{}{}
\newcommand{\KeywordTok}[1]{\textcolor[rgb]{0.00,0.44,0.13}{\textbf{{#1}}}}
\newcommand{\DataTypeTok}[1]{\textcolor[rgb]{0.56,0.13,0.00}{{#1}}}
\newcommand{\DecValTok}[1]{\textcolor[rgb]{0.25,0.63,0.44}{{#1}}}
\newcommand{\BaseNTok}[1]{\textcolor[rgb]{0.25,0.63,0.44}{{#1}}}
\newcommand{\FloatTok}[1]{\textcolor[rgb]{0.25,0.63,0.44}{{#1}}}
\newcommand{\CharTok}[1]{\textcolor[rgb]{0.25,0.44,0.63}{{#1}}}
\newcommand{\StringTok}[1]{\textcolor[rgb]{0.25,0.44,0.63}{{#1}}}
\newcommand{\CommentTok}[1]{\textcolor[rgb]{0.38,0.63,0.69}{\textit{{#1}}}}
\newcommand{\OtherTok}[1]{\textcolor[rgb]{0.00,0.44,0.13}{{#1}}}
\newcommand{\AlertTok}[1]{\textcolor[rgb]{1.00,0.00,0.00}{\textbf{{#1}}}}
\newcommand{\FunctionTok}[1]{\textcolor[rgb]{0.02,0.16,0.49}{{#1}}}
\newcommand{\RegionMarkerTok}[1]{{#1}}
\newcommand{\ErrorTok}[1]{\textcolor[rgb]{1.00,0.00,0.00}{\textbf{{#1}}}}
\newcommand{\NormalTok}[1]{{#1}}
\ifxetex
  \usepackage[setpagesize=false, % page size defined by xetex
              unicode=false, % unicode breaks when used with xetex
              xetex]{hyperref}
\else
  \usepackage[unicode=true]{hyperref}
\fi
\hypersetup{breaklinks=true,
            bookmarks=true,
            pdfauthor={},
            pdftitle={},
            colorlinks=true,
            citecolor=blue,
            urlcolor=blue,
            linkcolor=magenta,
            pdfborder={0 0 0}}
\urlstyle{same}  % don't use monospace font for urls
\setlength{\parindent}{0pt}
\setlength{\parskip}{6pt plus 2pt minus 1pt}
\setlength{\emergencystretch}{3em}  % prevent overfull lines
\setcounter{secnumdepth}{0}

%%% Use protect on footnotes to avoid problems with footnotes in titles
\let\rmarkdownfootnote\footnote%
\def\footnote{\protect\rmarkdownfootnote}

%%% Change title format to be more compact
\usepackage{titling}

% Create subtitle command for use in maketitle
\newcommand{\subtitle}[1]{
  \posttitle{
    \begin{center}\large#1\end{center}
    }
}

\setlength{\droptitle}{-2em}
  \title{}
  \pretitle{\vspace{\droptitle}}
  \posttitle{}
  \author{}
  \preauthor{}\postauthor{}
  \date{}
  \predate{}\postdate{}

\usepackage{polski}
\usepackage[T1]{fontenc}
\usepackage[utf8]{inputenc} 
%\usepackage[top=1.5cm, bottom=1.5cm, left=0.85cm, right=0.85cm]{geometry}
\usepackage{fancyhdr}
\pagestyle{fancy}
\fancyhead[RO,LE]{\bfseries \small{P. Auguścik, M. Kosiński, B. Sozańska, A. Szewczyk}}
\fancyhead[RE,LO]{\bfseries \small{Biostatystyka, Projekt nr 4}}
\AtBeginDocument{\thispagestyle{fancy}}
\usepackage{rotating}
\usepackage{subfigure}
\usepackage{pdflscape}
\usepackage{amsfonts}
\usepackage{amsmath}
\usepackage{amssymb}
\usepackage{color}
\usepackage{amsthm}
\usepackage{longtable}
\usepackage{wrapfig,booktabs}
\usepackage{tikz}
\usepackage{float}
\usepackage{hyperref} %pakiet do dodawania hiperłącz
\hypersetup{colorlinks=true,
            linkcolor=black,
            citecolor=black,
            urlcolor=black}
%\title{\textbf{\LARGE{Biostatystyka - Projekt zaliczeniowy} }}


\begin{document}

\maketitle


\thispagestyle{fancy} Analizie poddano pacjentów dotknietych dwoma
typami bialaczki: ostra i przewlekla. Dla pacjentów mierzono czas do
wystapienia pierwszego ze zdarzen: czas do nawrotu choroby, czas do
pojawienia sie symptomów przewleklego odrzutu przeszczepu oraz czas
przezycia.

\vspace{10pt}

\textbf{Sub-dystrybuanty.}

Chcac zbadac wplyw poszczególnych zmiennych dyskretnych na czas do
wystapienia zdarzenia sporzadzono wykresy sub-dystrybuant. Dla
poszczególnych typów zdarzen oszacowania sub-dystrybuant wygladaja
nastepujaco:

\begin{itemize}
\itemsep1pt\parskip0pt\parsep0pt
\item
  Dla metody pobierania komórek do przeszczepu
\end{itemize}

\vspace{-22pt}

\includegraphics[width=16cm,height=8.2cm]{plot1.pdf}

\begin{itemize}
\itemsep1pt\parskip0pt\parsep0pt
\item
  Dla typu bialaczki
\end{itemize}

\vspace{-22pt}

\includegraphics[width=16cm,height=8.2cm]{plot2.pdf} \newpage

\textbf{Test Graya}

Formalnie przy uzyciu testu Graya mozna sprawdzic czy róznice w
oszacowanych sub-dystrybuantach sa istotne statystycznie w podziale na
podgrupy ze wzgledu na zmienne:

\begin{itemize}
\itemsep1pt\parskip0pt\parsep0pt
\item
  Dla metod pobierania komórek do przeszczepu:
\end{itemize}

\begin{verbatim}
       stat        pv df
1 2.0192451 0.1553163  1
2 0.2114488 0.6456342  1
3 1.7635527 0.1841820  1
\end{verbatim}

\begin{itemize}
\itemsep1pt\parskip0pt\parsep0pt
\item
  Dla typu bialaczki:
\end{itemize}

\begin{verbatim}
        stat         pv df
1 5.34343189 0.02080048  1
2 3.71906498 0.05379449  1
3 0.01090047 0.91684770  1
\end{verbatim}

Z podsumowania testów widac, ze zachodza statystycznie istotne róznice w
oszacowanych sub-dystrybuantach dla typu bialaczki dla pierwszego typu
zdarzenia czyli odrzutu, na zakladanym poziomie istotnosci
\(\alpha=0.05\) (wartosc krytyczna testu jest równa \(0.02080048\)).
Mozna to równiez zaobserwowac na ponizszym wykresie prezentujacym
sub-dystrybuanty dla typu zdarzenia jakim jest odrzut, z podzialem na
typ bialaczki

\begin{center}\includegraphics{figure/beamer-unnamed-chunk-8-1} \end{center}

Widac, ze oszacowana sub-dystrybuanta dla \emph{ostrej} bialaczki jest
ponizej oszacowanej sub-dystrybuanty dla \emph{przewleklego} typu tej
choroby, co oznacza, ze pacjenci z \emph{ostra} bialaczka maja dluzsze
czasy do zdarzenia jakim jest odrzut przeszczepu.

\newpage
\textbf{Modele proporcjonalnych hazardów}

Dla danych dotyczacych wieku pacjenta, typu bialaczki i metody pobrania
komórek do przeszczepu dopasowujemy model PH dla funkcji hazardów
`specyficznych dla typów'. Podsumownia modeli dla typów zdarzen: odrzut,
nawrót, zgon, sa ponizej.

\begin{verbatim}
Call:
coxph(formula = Surv(first_t, first_e == 1) ~ diag + trt + age, 
    data = dane.red)

        coef exp(coef) se(coef)      z     p
diag  0.4712      1.60   0.2368  1.990 0.047
trt   0.0602      1.06   0.2201  0.274 0.780
age  -0.0104      0.99   0.0103 -1.006 0.310
\end{verbatim}

\begin{verbatim}
Call:
coxph(formula = Surv(first_t, first_e == 2) ~ diag + trt + age, 
    data = dane.red)

        coef exp(coef) se(coef)       z     p
diag -1.9448     0.143   1.1361 -1.7118 0.087
trt  -0.0800     0.923   0.9689 -0.0825 0.930
age  -0.0484     0.953   0.0433 -1.1182 0.260
\end{verbatim}

\begin{verbatim}
Call:
coxph(formula = Surv(first_t, first_e == 3) ~ diag + trt + age, 
    data = dane.red)

        coef exp(coef) se(coef)      z     p
diag -0.3965     0.673   0.9735 -0.407 0.680
trt  -1.5046     0.222   1.1522 -1.306 0.190
age   0.0973     1.102   0.0588  1.655 0.098
\end{verbatim}

Na badanym, zakladanym poziomie istotnosci \(\alpha=0.05\),
statystycznie istotnie rózny od 0 jest wspólczynnik przy zmiennej
\textsf{diag} odpowiadajacej typowi bialaczki w modelu dla zdarzenia
jakim jest nawrót choroby - wartosc krytyczna testu wyniosla
\(0.047<0.05\). Bialaczka przewlekla ma o 60\% wiekszy hazard
`specyficzny dla typu' na odrzut. Dla pozostalych zmiennych dla tego
typu zdarzenia oraz wszystkich zmiennych w pozostalych typach zdarzenia,
nie ma statystycznie istotnych podstaw aby odrzucic hipoteze zerowa
mówiaca o tym, ze wspólczynnik w modelu jest równy 0.

Sporzadzono równiez model hazardu sub-dystrybuanty, którego podsumowanie
wyglada nastepujaco

\begin{verbatim}
convergence:  TRUE 
coefficients:
    diag      trt      age 
 0.53950  0.31860 -0.01188 
standard errors:
[1] 0.228400 0.207900 0.009176
two-sided p-values:
 diag   trt   age 
0.018 0.130 0.200 
\end{verbatim}

\begin{verbatim}
convergence:  TRUE 
coefficients:
    diag      trt      age 
-1.91300 -0.12670 -0.04486 
standard errors:
[1] 1.2150 0.9466 0.0337
two-sided p-values:
diag  trt  age 
0.12 0.89 0.18 
\end{verbatim}

\begin{verbatim}
convergence:  TRUE 
coefficients:
    diag      trt      age 
-0.38580 -1.50400  0.09887 
standard errors:
[1] 0.7938 1.1230 0.0513
two-sided p-values:
 diag   trt   age 
0.630 0.180 0.054 
\end{verbatim}

Podobnie w modelu hazardu dla sub-dystrybuanty, jedyna zmienna istotnie
statystycznie rózna od 0 jest zmienna \textsf{diag} dla modelu dla typu
zdarzenia jakim jest nawrót, której wartosc krytyczna testu
\(0.018<0.05\) jest mniejsza od zakladanego poziomu istonotsci. Hazrad
dla sub-dystrybuanty dla pacjenta z przewlekla bialaczka jest o 71,5 \%
wiekszy niz dla pacjenta z ostra bialaczka. Dla pozostalych zmiennych w
tym modelu i dla wszystkich zmiennych w pozostalych modelach dla tych
zmiennych nie ma statystycznie istotnych podstaw by odrzucic hipotezy
zerowe o tym, ze wspólczynniki sa równe 0.

\textcolor{blue}{ooo i wlasnie nie mamy 12-tki: }

\begin{Shaded}
\begin{Highlighting}[]
\CommentTok{#12. Konstruujemy oszacowania funkcji skumulowanych czestosci odpowiadajacych modelowi PH dla #funkcji hazardu sub-dystrybuanty dla odrzutow}
\NormalTok{odrzut.pred <-}\StringTok{ }\KeywordTok{predict}\NormalTok{(mod.odrzut,}\DataTypeTok{cov1=}\KeywordTok{rbind}\NormalTok{(}\DecValTok{0}\NormalTok{,}\DecValTok{1}\NormalTok{))}
\KeywordTok{plot}\NormalTok{(odrzut.pred,}\DataTypeTok{lty=}\DecValTok{1}\NormalTok{:}\DecValTok{4}\NormalTok{,}\DataTypeTok{color=}\DecValTok{1}\NormalTok{:}\DecValTok{4}\NormalTok{)}
\KeywordTok{text}\NormalTok{(}\FloatTok{0.6}\NormalTok{, .}\DecValTok{275}\NormalTok{, }\StringTok{"WW, Odrzut"}\NormalTok{, }\DataTypeTok{col=}\DecValTok{1}\NormalTok{)}
\KeywordTok{text}\NormalTok{(}\FloatTok{0.8}\NormalTok{, .}\DecValTok{15}\NormalTok{, }\StringTok{"WM, Odrzut"}\NormalTok{, }\DataTypeTok{col=}\DecValTok{2}\NormalTok{)}
\end{Highlighting}
\end{Shaded}

\begin{center}\includegraphics{figure/beamer-unnamed-chunk-11-1} \end{center}

\newpage
\textbf{Kody?}

\end{document}
