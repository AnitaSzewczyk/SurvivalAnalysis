\documentclass[]{article}
\usepackage{lmodern}
\usepackage{amssymb,amsmath}
\usepackage{ifxetex,ifluatex}
\usepackage{fixltx2e} % provides \textsubscript
\ifnum 0\ifxetex 1\fi\ifluatex 1\fi=0 % if pdftex
  \usepackage[T1]{fontenc}
  \usepackage[utf8]{inputenc}
\else % if luatex or xelatex
  \ifxetex
    \usepackage{mathspec}
    \usepackage{xltxtra,xunicode}
  \else
    \usepackage{fontspec}
  \fi
  \defaultfontfeatures{Mapping=tex-text,Scale=MatchLowercase}
  \newcommand{\euro}{€}
\fi
% use upquote if available, for straight quotes in verbatim environments
\IfFileExists{upquote.sty}{\usepackage{upquote}}{}
% use microtype if available
\IfFileExists{microtype.sty}{%
\usepackage{microtype}
\UseMicrotypeSet[protrusion]{basicmath} % disable protrusion for tt fonts
}{}
\usepackage[margin=1in]{geometry}
\usepackage{color}
\usepackage{fancyvrb}
\newcommand{\VerbBar}{|}
\newcommand{\VERB}{\Verb[commandchars=\\\{\}]}
\DefineVerbatimEnvironment{Highlighting}{Verbatim}{commandchars=\\\{\}}
% Add ',fontsize=\small' for more characters per line
\newenvironment{Shaded}{}{}
\newcommand{\KeywordTok}[1]{\textcolor[rgb]{0.00,0.44,0.13}{\textbf{{#1}}}}
\newcommand{\DataTypeTok}[1]{\textcolor[rgb]{0.56,0.13,0.00}{{#1}}}
\newcommand{\DecValTok}[1]{\textcolor[rgb]{0.25,0.63,0.44}{{#1}}}
\newcommand{\BaseNTok}[1]{\textcolor[rgb]{0.25,0.63,0.44}{{#1}}}
\newcommand{\FloatTok}[1]{\textcolor[rgb]{0.25,0.63,0.44}{{#1}}}
\newcommand{\CharTok}[1]{\textcolor[rgb]{0.25,0.44,0.63}{{#1}}}
\newcommand{\StringTok}[1]{\textcolor[rgb]{0.25,0.44,0.63}{{#1}}}
\newcommand{\CommentTok}[1]{\textcolor[rgb]{0.38,0.63,0.69}{\textit{{#1}}}}
\newcommand{\OtherTok}[1]{\textcolor[rgb]{0.00,0.44,0.13}{{#1}}}
\newcommand{\AlertTok}[1]{\textcolor[rgb]{1.00,0.00,0.00}{\textbf{{#1}}}}
\newcommand{\FunctionTok}[1]{\textcolor[rgb]{0.02,0.16,0.49}{{#1}}}
\newcommand{\RegionMarkerTok}[1]{{#1}}
\newcommand{\ErrorTok}[1]{\textcolor[rgb]{1.00,0.00,0.00}{\textbf{{#1}}}}
\newcommand{\NormalTok}[1]{{#1}}
\ifxetex
  \usepackage[setpagesize=false, % page size defined by xetex
              unicode=false, % unicode breaks when used with xetex
              xetex]{hyperref}
\else
  \usepackage[unicode=true]{hyperref}
\fi
\hypersetup{breaklinks=true,
            bookmarks=true,
            pdfauthor={},
            pdftitle={},
            colorlinks=true,
            citecolor=blue,
            urlcolor=blue,
            linkcolor=magenta,
            pdfborder={0 0 0}}
\urlstyle{same}  % don't use monospace font for urls
\setlength{\parindent}{0pt}
\setlength{\parskip}{6pt plus 2pt minus 1pt}
\setlength{\emergencystretch}{3em}  % prevent overfull lines
\setcounter{secnumdepth}{0}

%%% Use protect on footnotes to avoid problems with footnotes in titles
\let\rmarkdownfootnote\footnote%
\def\footnote{\protect\rmarkdownfootnote}

%%% Change title format to be more compact
\usepackage{titling}
\setlength{\droptitle}{-2em}
  \title{}
  \pretitle{\vspace{\droptitle}}
  \posttitle{}
  \author{}
  \preauthor{}\postauthor{}
  \date{}
  \predate{}\postdate{}


\usepackage{polski}
\usepackage[T1]{fontenc}
\usepackage[utf8]{inputenc} 
%\usepackage[top=1.5cm, bottom=1.5cm, left=0.85cm, right=0.85cm]{geometry}
\usepackage{fancyhdr}
\pagestyle{fancy}
\fancyhead[RO,LE]{\bfseries \small{P. Auguścik, M. Kosiński, B. Sozańska, A. Szewczyk}}
\fancyhead[RE,LO]{\bfseries \small{Biostatystyka, Projekt nr 3}}
\AtBeginDocument{\thispagestyle{fancy}}
\usepackage{rotating}
\usepackage{subfigure}
\usepackage{pdflscape}
\usepackage{amsfonts}
\usepackage{amsmath}
\usepackage{amssymb}
\usepackage{color}
\usepackage{amsthm}
\usepackage{longtable}
\usepackage{wrapfig,booktabs}
\usepackage{tikz}
\usepackage{float}
\usepackage{hyperref} %pakiet do dodawania hiperłącz
\hypersetup{colorlinks=true,
            linkcolor=black,
            citecolor=black,
            urlcolor=black}
%\title{\textbf{\LARGE{Biostatystyka - Projekt zaliczeniowy} }}


\begin{document}

\maketitle


\thispagestyle{fancy}
\textbf{Zadanie 1 - czas do odtworzenia bariery krew-mleko u krów z zapaleniem wymienia}

Oceniamy efekt leczenia z uwzględnieniem ,,cielności'' krów z zapaleniem
wymienia. Dokonujemy analizy następujących zmiennych:

\begin{itemize}
\item cowid -- id krowy,
\item time -- czas do odtworzenia bariery (w dniach),
\item status -- wskaźnik odtworzenia bariery (0 –- nie, 1 -- tak),
\item drug -- leczenie (0 –- placebo, 1 -- aktywny lek),
\item heifer -- wskaźnik „cielności” (0 –- dwa lub więcej cielęta, 1 –- jedno lub
jałówka).
\end{itemize}

W związku z tym, że posiadamy po dwa pomiary czasów do odtworzenia
bariery dla każdej krowy (oddzielnie dla wymion leczonych i
nieleczonych), proponujemy model proporcjonalnych hazardów z
uwzględnieniem skorelowanych czasów tych zdarzeń.

\begin{verbatim}
Call:
coxph(formula = Surv(Time, Status) ~ Drug + Heifer + cluster(Cowid), 
    data = dane)

  n= 200, number of events= 162 

         coef exp(coef) se(coef) robust se     z Pr(>|z|)  
Drug   0.3371    1.4008   0.1577    0.1370 2.460   0.0139 *
Heifer 0.2261    1.2536   0.1703    0.2076 1.089   0.2761  
---
Signif. codes:  0 '***' 0.001 '**' 0.01 '*' 0.05 '.' 0.1 ' ' 1

       exp(coef) exp(-coef) lower .95 upper .95
Drug       1.401     0.7139    1.0709     1.832
Heifer     1.254     0.7977    0.8346     1.883

Concordance= 0.572  (se = 0.023 )
Rsquare= 0.032   (max possible= 1 )
Likelihood ratio test= 6.44  on 2 df,   p=0.03996
Wald test            = 7.28  on 2 df,   p=0.0263
Score (logrank) test = 6.55  on 2 df,   p=0.03779,   Robust = 7.03  p=0.02969

  (Note: the likelihood ratio and score tests assume independence of
     observations within a cluster, the Wald and robust score tests do not).
\end{verbatim}

Na poziomie istotności \(\alpha\) = 0.05 wskaźnik leczenia krów ma
istotny wpływ na szybkość odbudowywania się bariery krew-mleko u krów,
natomiast ,,cielność'' nie ma istotnego wpływu. Hazard odtworzenia
bariery zwiększa się 1.4 raza przy zastosowaniu leczenia względem
podawania placebo. Oznacza to, że czas \text{do odtworzenia} bariery
krew-mleko jest krótszy w przypadku zastosowania leczenia. Odchyelnie
estymatora odpornego jest mniejsze od odchylenia normalnego estymatora
współczynnika dla leczenia, co może świadczyć o tym, że uwzględnienie
skorelownia czasów jest słusznym zabiegiem. Jedynie test Walda nie
zakłada niezależności czasów, zatem na jego podstawie stwierdzamy, że
dopasowany przez nas model jest lepszy niż model pusty (p-wartość w
teście Walda wynosi 0.0263 oznacza, że na poziomie istotności \(0.05\)
mamy statystycznie istotne podstawy do odrzucenia hipotezy zerowej o
tym, że model pusty lepiej wyjaśnia dane).

\newpage
\textbf{Zadanie 2 - brakowanie jałówek}

Oceniamy czas do brakowania z uwzględnieniem efektu czasu oceny SCC i
log(SCC). Dokonujemy analizy następujących zmiennych:

\begin{itemize}
\item cowid -- id jałówki
\item time -- czas do brakowania (w dniach)
\item status -- wskaźnik brakowania (0 –- nie, 1 -- tak)
\item herd -- id stada
\item timeassess -- czas badania laboratoryjnego SCC (w dniach)
\item logSCC -- logarytm wartosci SCC
\end{itemize}

W związku z tym, że zasugerowano, iż strategia brakowania jałówek,
podobnie jak poziom SCC w mleku, mogą znacząco różnić się pomiędzy
stadami, zdecydowaliśmy się na model z podatnościami.

\begin{verbatim}
Call:
coxph(formula = Surv(Time, Status) ~ Timeassess + LogSCC + frailty(Herd), 
    data = data)

  n= 13835, number of events= 2729 

              coef     se(coef) se2      Chisq  DF    p      
Timeassess    0.002466 0.006951 0.006868   0.13   1.0 7.2e-01
LogSCC        0.069244 0.015549 0.015317  19.83   1.0 8.5e-06
frailty(Herd)                            357.62 309.7 3.1e-02

           exp(coef) exp(-coef) lower .95 upper .95
Timeassess     1.002     0.9975    0.9889     1.016
LogSCC         1.072     0.9331    1.0395     1.105

Iterations: 8 outer, 30 Newton-Raphson
     Variance of random effect= 0.1302054   I-likelihood = -25493.5 
Degrees of freedom for terms=   1.0   1.0 309.7 
Concordance= 0.742  (se = 0.006 )
Likelihood ratio test= 661.7  on 311.7 df,   p=0
\end{verbatim}

Na poziomie istotności \(\alpha\) = 0.05, zmienna \texttt{LogSCC} jest
statystycznie istotna w modelu. Wartość krytyczna przy zmiennej
\texttt{frailty(Herd)}, która jest mniejsza od zakładanego poziomu
istotności \(0.05\) oznacza, że mamy statystycznie istotne podstawy do
stwierdzenia, że istenieje efekt stada. Przy wzroście logarytmu wartości
SCC o jeden, hazard do brakowania jałówki wzrasta 1.072 raza. Wymienione
efekty odnoszą się tylko do krów z tego samego stada, pomiędzy stadami
mogą zachodzić inne zależności.

\newpage
\textbf{Kody:}

\begin{Shaded}
\begin{Highlighting}[]
\NormalTok{## Zadanie 1.}
\KeywordTok{library}\NormalTok{(survival)}
\NormalTok{dane <-}\StringTok{ }\KeywordTok{read.table}\NormalTok{(}\StringTok{"C:}\CharTok{\textbackslash{}\textbackslash{}}\StringTok{SurvivalAnalysis}\CharTok{\textbackslash{}\textbackslash{}}\StringTok{PD3}\CharTok{\textbackslash{}\textbackslash{}}\StringTok{reconstitution.dat"}\NormalTok{, }\DataTypeTok{header=}\OtherTok{TRUE}\NormalTok{, }\DataTypeTok{sep=}\StringTok{","}\NormalTok{)}
\NormalTok{model <-}\StringTok{ }\KeywordTok{coxph}\NormalTok{(}\KeywordTok{Surv}\NormalTok{(Time, Status)~Drug +}\StringTok{ }\NormalTok{Heifer +}\StringTok{ }\KeywordTok{cluster}\NormalTok{(Cowid), }\DataTypeTok{data =} \NormalTok{dane) }
\KeywordTok{summary}\NormalTok{(model)}

\NormalTok{## Zadanie 2.}
\NormalTok{data <-}\StringTok{ }\KeywordTok{read.table}\NormalTok{(}\StringTok{"C:}\CharTok{\textbackslash{}\textbackslash{}}\StringTok{SurvivalAnalysis}\CharTok{\textbackslash{}\textbackslash{}}\StringTok{PD3}\CharTok{\textbackslash{}\textbackslash{}}\StringTok{culling.dat"}\NormalTok{, }\DataTypeTok{header=}\OtherTok{TRUE}\NormalTok{, }\DataTypeTok{sep=}\StringTok{","}\NormalTok{)}
\NormalTok{frailty <-}\StringTok{ }\KeywordTok{coxph}\NormalTok{(}\KeywordTok{Surv}\NormalTok{(Time, Status)~Timeassess +}\StringTok{ }\NormalTok{LogSCC +}\StringTok{ }\KeywordTok{frailty}\NormalTok{(Herd), }\DataTypeTok{data =} \NormalTok{data)}
\KeywordTok{summary}\NormalTok{(frailty)}
\end{Highlighting}
\end{Shaded}

\end{document}
